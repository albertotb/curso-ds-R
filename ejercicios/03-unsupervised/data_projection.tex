% Options for packages loaded elsewhere
\PassOptionsToPackage{unicode}{hyperref}
\PassOptionsToPackage{hyphens}{url}
%
\documentclass[
]{article}
\usepackage{lmodern}
\usepackage{amssymb,amsmath}
\usepackage{ifxetex,ifluatex}
\ifnum 0\ifxetex 1\fi\ifluatex 1\fi=0 % if pdftex
  \usepackage[T1]{fontenc}
  \usepackage[utf8]{inputenc}
  \usepackage{textcomp} % provide euro and other symbols
\else % if luatex or xetex
  \usepackage{unicode-math}
  \defaultfontfeatures{Scale=MatchLowercase}
  \defaultfontfeatures[\rmfamily]{Ligatures=TeX,Scale=1}
\fi
% Use upquote if available, for straight quotes in verbatim environments
\IfFileExists{upquote.sty}{\usepackage{upquote}}{}
\IfFileExists{microtype.sty}{% use microtype if available
  \usepackage[]{microtype}
  \UseMicrotypeSet[protrusion]{basicmath} % disable protrusion for tt fonts
}{}
\makeatletter
\@ifundefined{KOMAClassName}{% if non-KOMA class
  \IfFileExists{parskip.sty}{%
    \usepackage{parskip}
  }{% else
    \setlength{\parindent}{0pt}
    \setlength{\parskip}{6pt plus 2pt minus 1pt}}
}{% if KOMA class
  \KOMAoptions{parskip=half}}
\makeatother
\usepackage{xcolor}
\IfFileExists{xurl.sty}{\usepackage{xurl}}{} % add URL line breaks if available
\IfFileExists{bookmark.sty}{\usepackage{bookmark}}{\usepackage{hyperref}}
\hypersetup{
  pdftitle={PCA para visualización de datos},
  pdfauthor={Victor Gallego y Roi Naveiro},
  hidelinks,
  pdfcreator={LaTeX via pandoc}}
\urlstyle{same} % disable monospaced font for URLs
\usepackage[margin=1in]{geometry}
\usepackage{color}
\usepackage{fancyvrb}
\newcommand{\VerbBar}{|}
\newcommand{\VERB}{\Verb[commandchars=\\\{\}]}
\DefineVerbatimEnvironment{Highlighting}{Verbatim}{commandchars=\\\{\}}
% Add ',fontsize=\small' for more characters per line
\usepackage{framed}
\definecolor{shadecolor}{RGB}{248,248,248}
\newenvironment{Shaded}{\begin{snugshade}}{\end{snugshade}}
\newcommand{\AlertTok}[1]{\textcolor[rgb]{0.94,0.16,0.16}{#1}}
\newcommand{\AnnotationTok}[1]{\textcolor[rgb]{0.56,0.35,0.01}{\textbf{\textit{#1}}}}
\newcommand{\AttributeTok}[1]{\textcolor[rgb]{0.77,0.63,0.00}{#1}}
\newcommand{\BaseNTok}[1]{\textcolor[rgb]{0.00,0.00,0.81}{#1}}
\newcommand{\BuiltInTok}[1]{#1}
\newcommand{\CharTok}[1]{\textcolor[rgb]{0.31,0.60,0.02}{#1}}
\newcommand{\CommentTok}[1]{\textcolor[rgb]{0.56,0.35,0.01}{\textit{#1}}}
\newcommand{\CommentVarTok}[1]{\textcolor[rgb]{0.56,0.35,0.01}{\textbf{\textit{#1}}}}
\newcommand{\ConstantTok}[1]{\textcolor[rgb]{0.00,0.00,0.00}{#1}}
\newcommand{\ControlFlowTok}[1]{\textcolor[rgb]{0.13,0.29,0.53}{\textbf{#1}}}
\newcommand{\DataTypeTok}[1]{\textcolor[rgb]{0.13,0.29,0.53}{#1}}
\newcommand{\DecValTok}[1]{\textcolor[rgb]{0.00,0.00,0.81}{#1}}
\newcommand{\DocumentationTok}[1]{\textcolor[rgb]{0.56,0.35,0.01}{\textbf{\textit{#1}}}}
\newcommand{\ErrorTok}[1]{\textcolor[rgb]{0.64,0.00,0.00}{\textbf{#1}}}
\newcommand{\ExtensionTok}[1]{#1}
\newcommand{\FloatTok}[1]{\textcolor[rgb]{0.00,0.00,0.81}{#1}}
\newcommand{\FunctionTok}[1]{\textcolor[rgb]{0.00,0.00,0.00}{#1}}
\newcommand{\ImportTok}[1]{#1}
\newcommand{\InformationTok}[1]{\textcolor[rgb]{0.56,0.35,0.01}{\textbf{\textit{#1}}}}
\newcommand{\KeywordTok}[1]{\textcolor[rgb]{0.13,0.29,0.53}{\textbf{#1}}}
\newcommand{\NormalTok}[1]{#1}
\newcommand{\OperatorTok}[1]{\textcolor[rgb]{0.81,0.36,0.00}{\textbf{#1}}}
\newcommand{\OtherTok}[1]{\textcolor[rgb]{0.56,0.35,0.01}{#1}}
\newcommand{\PreprocessorTok}[1]{\textcolor[rgb]{0.56,0.35,0.01}{\textit{#1}}}
\newcommand{\RegionMarkerTok}[1]{#1}
\newcommand{\SpecialCharTok}[1]{\textcolor[rgb]{0.00,0.00,0.00}{#1}}
\newcommand{\SpecialStringTok}[1]{\textcolor[rgb]{0.31,0.60,0.02}{#1}}
\newcommand{\StringTok}[1]{\textcolor[rgb]{0.31,0.60,0.02}{#1}}
\newcommand{\VariableTok}[1]{\textcolor[rgb]{0.00,0.00,0.00}{#1}}
\newcommand{\VerbatimStringTok}[1]{\textcolor[rgb]{0.31,0.60,0.02}{#1}}
\newcommand{\WarningTok}[1]{\textcolor[rgb]{0.56,0.35,0.01}{\textbf{\textit{#1}}}}
\usepackage{graphicx,grffile}
\makeatletter
\def\maxwidth{\ifdim\Gin@nat@width>\linewidth\linewidth\else\Gin@nat@width\fi}
\def\maxheight{\ifdim\Gin@nat@height>\textheight\textheight\else\Gin@nat@height\fi}
\makeatother
% Scale images if necessary, so that they will not overflow the page
% margins by default, and it is still possible to overwrite the defaults
% using explicit options in \includegraphics[width, height, ...]{}
\setkeys{Gin}{width=\maxwidth,height=\maxheight,keepaspectratio}
% Set default figure placement to htbp
\makeatletter
\def\fps@figure{htbp}
\makeatother
\setlength{\emergencystretch}{3em} % prevent overfull lines
\providecommand{\tightlist}{%
  \setlength{\itemsep}{0pt}\setlength{\parskip}{0pt}}
\setcounter{secnumdepth}{-\maxdimen} % remove section numbering

\title{PCA para visualización de datos}
\author{Victor Gallego y Roi Naveiro}
\date{01/04/2019}

\begin{document}
\maketitle

\hypertarget{funciones-auxiliares}{%
\subsection{Funciones auxiliares}\label{funciones-auxiliares}}

\begin{itemize}
\tightlist
\item
  show\_digit: Hace una gráfica del dígito en cuestión.
\item
  load\_image\_file: Para cargar las imágenes de los dígitos
\item
  load\_label\_file: Para cargar las etiquetas
\end{itemize}

\begin{Shaded}
\begin{Highlighting}[]
\NormalTok{show_digit =}\StringTok{ }\ControlFlowTok{function}\NormalTok{(arr784, }\DataTypeTok{col =} \KeywordTok{gray}\NormalTok{(}\DecValTok{12}\OperatorTok{:}\DecValTok{1} \OperatorTok{/}\StringTok{ }\DecValTok{12}\NormalTok{), ...) \{}
  \KeywordTok{image}\NormalTok{(}\KeywordTok{matrix}\NormalTok{(}\KeywordTok{as.matrix}\NormalTok{(arr784[}\OperatorTok{-}\DecValTok{785}\NormalTok{]), }\DataTypeTok{nrow =} \DecValTok{28}\NormalTok{)[, }\DecValTok{28}\OperatorTok{:}\DecValTok{1}\NormalTok{], }\DataTypeTok{col =}\NormalTok{ col, ...)}
\NormalTok{\}}

\NormalTok{load_image_file =}\StringTok{ }\ControlFlowTok{function}\NormalTok{(filename) \{}
\NormalTok{  ret =}\StringTok{ }\KeywordTok{list}\NormalTok{()}
\NormalTok{  f =}\StringTok{ }\KeywordTok{file}\NormalTok{(filename, }\StringTok{'rb'}\NormalTok{)}
  \KeywordTok{readBin}\NormalTok{(f, }\StringTok{'integer'}\NormalTok{, }\DataTypeTok{n =} \DecValTok{1}\NormalTok{, }\DataTypeTok{size =} \DecValTok{4}\NormalTok{, }\DataTypeTok{endian =} \StringTok{'big'}\NormalTok{)}
\NormalTok{  n    =}\StringTok{ }\KeywordTok{readBin}\NormalTok{(f, }\StringTok{'integer'}\NormalTok{, }\DataTypeTok{n =} \DecValTok{1}\NormalTok{, }\DataTypeTok{size =} \DecValTok{4}\NormalTok{, }\DataTypeTok{endian =} \StringTok{'big'}\NormalTok{)}
\NormalTok{  nrow =}\StringTok{ }\KeywordTok{readBin}\NormalTok{(f, }\StringTok{'integer'}\NormalTok{, }\DataTypeTok{n =} \DecValTok{1}\NormalTok{, }\DataTypeTok{size =} \DecValTok{4}\NormalTok{, }\DataTypeTok{endian =} \StringTok{'big'}\NormalTok{)}
\NormalTok{  ncol =}\StringTok{ }\KeywordTok{readBin}\NormalTok{(f, }\StringTok{'integer'}\NormalTok{, }\DataTypeTok{n =} \DecValTok{1}\NormalTok{, }\DataTypeTok{size =} \DecValTok{4}\NormalTok{, }\DataTypeTok{endian =} \StringTok{'big'}\NormalTok{)}
\NormalTok{  x =}\StringTok{ }\KeywordTok{readBin}\NormalTok{(f, }\StringTok{'integer'}\NormalTok{, }\DataTypeTok{n =}\NormalTok{ n }\OperatorTok{*}\StringTok{ }\NormalTok{nrow }\OperatorTok{*}\StringTok{ }\NormalTok{ncol, }\DataTypeTok{size =} \DecValTok{1}\NormalTok{, }\DataTypeTok{signed =} \OtherTok{FALSE}\NormalTok{)}
  \KeywordTok{close}\NormalTok{(f)}
  \KeywordTok{data.frame}\NormalTok{(}\KeywordTok{matrix}\NormalTok{(x, }\DataTypeTok{ncol =}\NormalTok{ nrow }\OperatorTok{*}\StringTok{ }\NormalTok{ncol, }\DataTypeTok{byrow =} \OtherTok{TRUE}\NormalTok{))}
\NormalTok{\}}

\NormalTok{load_label_file =}\StringTok{ }\ControlFlowTok{function}\NormalTok{(filename) \{}
\NormalTok{  f =}\StringTok{ }\KeywordTok{file}\NormalTok{(filename, }\StringTok{'rb'}\NormalTok{)}
  \KeywordTok{readBin}\NormalTok{(f, }\StringTok{'integer'}\NormalTok{, }\DataTypeTok{n =} \DecValTok{1}\NormalTok{, }\DataTypeTok{size =} \DecValTok{4}\NormalTok{, }\DataTypeTok{endian =} \StringTok{'big'}\NormalTok{)}
\NormalTok{  n =}\StringTok{ }\KeywordTok{readBin}\NormalTok{(f, }\StringTok{'integer'}\NormalTok{, }\DataTypeTok{n =} \DecValTok{1}\NormalTok{, }\DataTypeTok{size =} \DecValTok{4}\NormalTok{, }\DataTypeTok{endian =} \StringTok{'big'}\NormalTok{)}
\NormalTok{  y =}\StringTok{ }\KeywordTok{readBin}\NormalTok{(f, }\StringTok{'integer'}\NormalTok{, }\DataTypeTok{n =}\NormalTok{ n, }\DataTypeTok{size =} \DecValTok{1}\NormalTok{, }\DataTypeTok{signed =} \OtherTok{FALSE}\NormalTok{)}
  \KeywordTok{close}\NormalTok{(f)}
\NormalTok{  y}
\NormalTok{\}}
\end{Highlighting}
\end{Shaded}

\hypertarget{lectura-de-datos}{%
\subsection{Lectura de Datos}\label{lectura-de-datos}}

Cargamos el dataset MNIST.

\begin{Shaded}
\begin{Highlighting}[]
\NormalTok{test  =}\StringTok{ }\KeywordTok{load_image_file}\NormalTok{(}\StringTok{"src/t10k-images.idx3-ubyte"}\NormalTok{)}
\NormalTok{test}\OperatorTok{$}\NormalTok{y  =}\StringTok{ }\KeywordTok{as.factor}\NormalTok{(}\KeywordTok{load_label_file}\NormalTok{(}\StringTok{"src/t10k-labels.idx1-ubyte"}\NormalTok{))}
\end{Highlighting}
\end{Shaded}

Esta base de datos consta de 10000 imágenes en escala de gris a 28 x 28,
de los dígitos del 0 al 9 (escritos a mano). Explora la base de datos y
visualiza algunos ejemplos usando la función show\_digit.

\begin{Shaded}
\begin{Highlighting}[]
\CommentTok{## CÓDIGO}
\end{Highlighting}
\end{Shaded}

\hypertarget{proyecciuxf3n-a-2d-usando-pca}{%
\subsection{Proyección a 2D usando
PCA}\label{proyecciuxf3n-a-2d-usando-pca}}

Usando el paquete prcomp de R base, proyecta las imágenes de los dígitos
a 2 dimensiones. Muestra una gráfica de los puntos proyectados,
etiquetando cada punto con el nombre del dígito correspondiente

\begin{Shaded}
\begin{Highlighting}[]
\CommentTok{## CÓDIGO}
\end{Highlighting}
\end{Shaded}

\hypertarget{pregunta}{%
\subsubsection{Pregunta}\label{pregunta}}

Pinta la curva de número de componentes frente a proporción de varianza
explicada varianza explicada. ¿Cuántas componentes son necesarias para
explicar el 99\% de la varianza?

\begin{Shaded}
\begin{Highlighting}[]
\CommentTok{## CÓDIGO}
\end{Highlighting}
\end{Shaded}

\hypertarget{proyecciuxf3n-a-2d-usando-t-sne}{%
\subsection{Proyección a 2D usando
t-SNE}\label{proyecciuxf3n-a-2d-usando-t-sne}}

Usando el paquete Rtsne, proyecta las imágenes de los dígitos a 2
dimensiones. Muestra una gráfica de los puntos proyectados, etiquetando
cada punto con el nombre del dígito correspondiente. ¿Observas algún
patrón?

\begin{Shaded}
\begin{Highlighting}[]
\CommentTok{# CÓDIGO}
\end{Highlighting}
\end{Shaded}

\end{document}
